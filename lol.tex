% Options for packages loaded elsewhere
\PassOptionsToPackage{unicode}{hyperref}
\PassOptionsToPackage{hyphens}{url}
%
\documentclass[
]{article}
\title{League of Legends Story Telling}
\author{}
\date{\vspace{-2.5em}}

\usepackage{amsmath,amssymb}
\usepackage{lmodern}
\usepackage{iftex}
\ifPDFTeX
  \usepackage[T1]{fontenc}
  \usepackage[utf8]{inputenc}
  \usepackage{textcomp} % provide euro and other symbols
\else % if luatex or xetex
  \usepackage{unicode-math}
  \defaultfontfeatures{Scale=MatchLowercase}
  \defaultfontfeatures[\rmfamily]{Ligatures=TeX,Scale=1}
\fi
% Use upquote if available, for straight quotes in verbatim environments
\IfFileExists{upquote.sty}{\usepackage{upquote}}{}
\IfFileExists{microtype.sty}{% use microtype if available
  \usepackage[]{microtype}
  \UseMicrotypeSet[protrusion]{basicmath} % disable protrusion for tt fonts
}{}
\makeatletter
\@ifundefined{KOMAClassName}{% if non-KOMA class
  \IfFileExists{parskip.sty}{%
    \usepackage{parskip}
  }{% else
    \setlength{\parindent}{0pt}
    \setlength{\parskip}{6pt plus 2pt minus 1pt}}
}{% if KOMA class
  \KOMAoptions{parskip=half}}
\makeatother
\usepackage{xcolor}
\IfFileExists{xurl.sty}{\usepackage{xurl}}{} % add URL line breaks if available
\IfFileExists{bookmark.sty}{\usepackage{bookmark}}{\usepackage{hyperref}}
\hypersetup{
  pdftitle={League of Legends Story Telling},
  hidelinks,
  pdfcreator={LaTeX via pandoc}}
\urlstyle{same} % disable monospaced font for URLs
\usepackage[margin=1in]{geometry}
\usepackage{color}
\usepackage{fancyvrb}
\newcommand{\VerbBar}{|}
\newcommand{\VERB}{\Verb[commandchars=\\\{\}]}
\DefineVerbatimEnvironment{Highlighting}{Verbatim}{commandchars=\\\{\}}
% Add ',fontsize=\small' for more characters per line
\usepackage{framed}
\definecolor{shadecolor}{RGB}{248,248,248}
\newenvironment{Shaded}{\begin{snugshade}}{\end{snugshade}}
\newcommand{\AlertTok}[1]{\textcolor[rgb]{0.94,0.16,0.16}{#1}}
\newcommand{\AnnotationTok}[1]{\textcolor[rgb]{0.56,0.35,0.01}{\textbf{\textit{#1}}}}
\newcommand{\AttributeTok}[1]{\textcolor[rgb]{0.77,0.63,0.00}{#1}}
\newcommand{\BaseNTok}[1]{\textcolor[rgb]{0.00,0.00,0.81}{#1}}
\newcommand{\BuiltInTok}[1]{#1}
\newcommand{\CharTok}[1]{\textcolor[rgb]{0.31,0.60,0.02}{#1}}
\newcommand{\CommentTok}[1]{\textcolor[rgb]{0.56,0.35,0.01}{\textit{#1}}}
\newcommand{\CommentVarTok}[1]{\textcolor[rgb]{0.56,0.35,0.01}{\textbf{\textit{#1}}}}
\newcommand{\ConstantTok}[1]{\textcolor[rgb]{0.00,0.00,0.00}{#1}}
\newcommand{\ControlFlowTok}[1]{\textcolor[rgb]{0.13,0.29,0.53}{\textbf{#1}}}
\newcommand{\DataTypeTok}[1]{\textcolor[rgb]{0.13,0.29,0.53}{#1}}
\newcommand{\DecValTok}[1]{\textcolor[rgb]{0.00,0.00,0.81}{#1}}
\newcommand{\DocumentationTok}[1]{\textcolor[rgb]{0.56,0.35,0.01}{\textbf{\textit{#1}}}}
\newcommand{\ErrorTok}[1]{\textcolor[rgb]{0.64,0.00,0.00}{\textbf{#1}}}
\newcommand{\ExtensionTok}[1]{#1}
\newcommand{\FloatTok}[1]{\textcolor[rgb]{0.00,0.00,0.81}{#1}}
\newcommand{\FunctionTok}[1]{\textcolor[rgb]{0.00,0.00,0.00}{#1}}
\newcommand{\ImportTok}[1]{#1}
\newcommand{\InformationTok}[1]{\textcolor[rgb]{0.56,0.35,0.01}{\textbf{\textit{#1}}}}
\newcommand{\KeywordTok}[1]{\textcolor[rgb]{0.13,0.29,0.53}{\textbf{#1}}}
\newcommand{\NormalTok}[1]{#1}
\newcommand{\OperatorTok}[1]{\textcolor[rgb]{0.81,0.36,0.00}{\textbf{#1}}}
\newcommand{\OtherTok}[1]{\textcolor[rgb]{0.56,0.35,0.01}{#1}}
\newcommand{\PreprocessorTok}[1]{\textcolor[rgb]{0.56,0.35,0.01}{\textit{#1}}}
\newcommand{\RegionMarkerTok}[1]{#1}
\newcommand{\SpecialCharTok}[1]{\textcolor[rgb]{0.00,0.00,0.00}{#1}}
\newcommand{\SpecialStringTok}[1]{\textcolor[rgb]{0.31,0.60,0.02}{#1}}
\newcommand{\StringTok}[1]{\textcolor[rgb]{0.31,0.60,0.02}{#1}}
\newcommand{\VariableTok}[1]{\textcolor[rgb]{0.00,0.00,0.00}{#1}}
\newcommand{\VerbatimStringTok}[1]{\textcolor[rgb]{0.31,0.60,0.02}{#1}}
\newcommand{\WarningTok}[1]{\textcolor[rgb]{0.56,0.35,0.01}{\textbf{\textit{#1}}}}
\usepackage{graphicx}
\makeatletter
\def\maxwidth{\ifdim\Gin@nat@width>\linewidth\linewidth\else\Gin@nat@width\fi}
\def\maxheight{\ifdim\Gin@nat@height>\textheight\textheight\else\Gin@nat@height\fi}
\makeatother
% Scale images if necessary, so that they will not overflow the page
% margins by default, and it is still possible to overwrite the defaults
% using explicit options in \includegraphics[width, height, ...]{}
\setkeys{Gin}{width=\maxwidth,height=\maxheight,keepaspectratio}
% Set default figure placement to htbp
\makeatletter
\def\fps@figure{htbp}
\makeatother
\setlength{\emergencystretch}{3em} % prevent overfull lines
\providecommand{\tightlist}{%
  \setlength{\itemsep}{0pt}\setlength{\parskip}{0pt}}
\setcounter{secnumdepth}{-\maxdimen} % remove section numbering
<!--radix_placeholder_navigation_in_header-->
<meta name="distill:offset" content=""/>

<script type="application/javascript">

  window.headroom_prevent_pin = false;

  window.document.addEventListener("DOMContentLoaded", function (event) {

    // initialize headroom for banner
    var header = $('header').get(0);
    var headerHeight = header.offsetHeight;
    var headroom = new Headroom(header, {
      tolerance: 5,
      onPin : function() {
        if (window.headroom_prevent_pin) {
          window.headroom_prevent_pin = false;
          headroom.unpin();
        }
      }
    });
    headroom.init();
    if(window.location.hash)
      headroom.unpin();
    $(header).addClass('headroom--transition');

    // offset scroll location for banner on hash change
    // (see: https://github.com/WickyNilliams/headroom.js/issues/38)
    window.addEventListener("hashchange", function(event) {
      window.scrollTo(0, window.pageYOffset - (headerHeight + 25));
    });

    // responsive menu
    $('.distill-site-header').each(function(i, val) {
      var topnav = $(this);
      var toggle = topnav.find('.nav-toggle');
      toggle.on('click', function() {
        topnav.toggleClass('responsive');
      });
    });

    // nav dropdowns
    $('.nav-dropbtn').click(function(e) {
      $(this).next('.nav-dropdown-content').toggleClass('nav-dropdown-active');
      $(this).parent().siblings('.nav-dropdown')
         .children('.nav-dropdown-content').removeClass('nav-dropdown-active');
    });
    $("body").click(function(e){
      $('.nav-dropdown-content').removeClass('nav-dropdown-active');
    });
    $(".nav-dropdown").click(function(e){
      e.stopPropagation();
    });
  });
</script>

<style type="text/css">

/* Theme (user-documented overrideables for nav appearance) */

.distill-site-nav {
  color: rgba(255, 255, 255, 0.8);
  background-color: #0F2E3D;
  font-size: 15px;
  font-weight: 300;
}

.distill-site-nav a {
  color: inherit;
  text-decoration: none;
}

.distill-site-nav a:hover {
  color: white;
}

@media print {
  .distill-site-nav {
    display: none;
  }
}

.distill-site-header {

}

.distill-site-footer {

}


/* Site Header */

.distill-site-header {
  width: 100%;
  box-sizing: border-box;
  z-index: 3;
}

.distill-site-header .nav-left {
  display: inline-block;
  margin-left: 8px;
}

@media screen and (max-width: 768px) {
  .distill-site-header .nav-left {
    margin-left: 0;
  }
}


.distill-site-header .nav-right {
  float: right;
  margin-right: 8px;
}

.distill-site-header a,
.distill-site-header .title {
  display: inline-block;
  text-align: center;
  padding: 14px 10px 14px 10px;
}

.distill-site-header .title {
  font-size: 18px;
  min-width: 150px;
}

.distill-site-header .logo {
  padding: 0;
}

.distill-site-header .logo img {
  display: none;
  max-height: 20px;
  width: auto;
  margin-bottom: -4px;
}

.distill-site-header .nav-image img {
  max-height: 18px;
  width: auto;
  display: inline-block;
  margin-bottom: -3px;
}



@media screen and (min-width: 1000px) {
  .distill-site-header .logo img {
    display: inline-block;
  }
  .distill-site-header .nav-left {
    margin-left: 20px;
  }
  .distill-site-header .nav-right {
    margin-right: 20px;
  }
  .distill-site-header .title {
    padding-left: 12px;
  }
}


.distill-site-header .nav-toggle {
  display: none;
}

.nav-dropdown {
  display: inline-block;
  position: relative;
}

.nav-dropdown .nav-dropbtn {
  border: none;
  outline: none;
  color: rgba(255, 255, 255, 0.8);
  padding: 16px 10px;
  background-color: transparent;
  font-family: inherit;
  font-size: inherit;
  font-weight: inherit;
  margin: 0;
  margin-top: 1px;
  z-index: 2;
}

.nav-dropdown-content {
  display: none;
  position: absolute;
  background-color: white;
  min-width: 200px;
  border: 1px solid rgba(0,0,0,0.15);
  border-radius: 4px;
  box-shadow: 0px 8px 16px 0px rgba(0,0,0,0.1);
  z-index: 1;
  margin-top: 2px;
  white-space: nowrap;
  padding-top: 4px;
  padding-bottom: 4px;
}

.nav-dropdown-content hr {
  margin-top: 4px;
  margin-bottom: 4px;
  border: none;
  border-bottom: 1px solid rgba(0, 0, 0, 0.1);
}

.nav-dropdown-active {
  display: block;
}

.nav-dropdown-content a, .nav-dropdown-content .nav-dropdown-header {
  color: black;
  padding: 6px 24px;
  text-decoration: none;
  display: block;
  text-align: left;
}

.nav-dropdown-content .nav-dropdown-header {
  display: block;
  padding: 5px 24px;
  padding-bottom: 0;
  text-transform: uppercase;
  font-size: 14px;
  color: #999999;
  white-space: nowrap;
}

.nav-dropdown:hover .nav-dropbtn {
  color: white;
}

.nav-dropdown-content a:hover {
  background-color: #ddd;
  color: black;
}

.nav-right .nav-dropdown-content {
  margin-left: -45%;
  right: 0;
}

@media screen and (max-width: 768px) {
  .distill-site-header a, .distill-site-header .nav-dropdown  {display: none;}
  .distill-site-header a.nav-toggle {
    float: right;
    display: block;
  }
  .distill-site-header .title {
    margin-left: 0;
  }
  .distill-site-header .nav-right {
    margin-right: 0;
  }
  .distill-site-header {
    overflow: hidden;
  }
  .nav-right .nav-dropdown-content {
    margin-left: 0;
  }
}


@media screen and (max-width: 768px) {
  .distill-site-header.responsive {position: relative; min-height: 500px; }
  .distill-site-header.responsive a.nav-toggle {
    position: absolute;
    right: 0;
    top: 0;
  }
  .distill-site-header.responsive a,
  .distill-site-header.responsive .nav-dropdown {
    display: block;
    text-align: left;
  }
  .distill-site-header.responsive .nav-left,
  .distill-site-header.responsive .nav-right {
    width: 100%;
  }
  .distill-site-header.responsive .nav-dropdown {float: none;}
  .distill-site-header.responsive .nav-dropdown-content {position: relative;}
  .distill-site-header.responsive .nav-dropdown .nav-dropbtn {
    display: block;
    width: 100%;
    text-align: left;
  }
}

/* Site Footer */

.distill-site-footer {
  width: 100%;
  overflow: hidden;
  box-sizing: border-box;
  z-index: 3;
  margin-top: 30px;
  padding-top: 30px;
  padding-bottom: 30px;
  text-align: center;
}

/* Headroom */

d-title {
  padding-top: 6rem;
}

@media print {
  d-title {
    padding-top: 4rem;
  }
}

.headroom {
  z-index: 1000;
  position: fixed;
  top: 0;
  left: 0;
  right: 0;
}

.headroom--transition {
  transition: all .4s ease-in-out;
}

.headroom--unpinned {
  top: -100px;
}

.headroom--pinned {
  top: 0;
}

/* adjust viewport for navbar height */
/* helps vertically center bootstrap (non-distill) content */
.min-vh-100 {
  min-height: calc(100vh - 100px) !important;
}

</style>

<script src="site_libs/jquery-3.6.0/jquery-3.6.0.min.js"></script>
<link href="site_libs/font-awesome-6.2.1/css/all.min.css" rel="stylesheet"/>
<link href="site_libs/font-awesome-6.2.1/css/v4-shims.min.css" rel="stylesheet"/>
<script src="site_libs/headroom-0.9.4/headroom.min.js"></script>
<script src="site_libs/autocomplete-0.37.1/autocomplete.min.js"></script>
<script src="site_libs/fuse-6.4.1/fuse.min.js"></script>

<script type="application/javascript">

function getMeta(metaName) {
  var metas = document.getElementsByTagName('meta');
  for (let i = 0; i < metas.length; i++) {
    if (metas[i].getAttribute('name') === metaName) {
      return metas[i].getAttribute('content');
    }
  }
  return '';
}

function offsetURL(url) {
  var offset = getMeta('distill:offset');
  return offset ? offset + '/' + url : url;
}

function createFuseIndex() {

  // create fuse index
  var options = {
    keys: [
      { name: 'title', weight: 20 },
      { name: 'categories', weight: 15 },
      { name: 'description', weight: 10 },
      { name: 'contents', weight: 5 },
    ],
    ignoreLocation: true,
    threshold: 0
  };
  var fuse = new window.Fuse([], options);

  // fetch the main search.json
  return fetch(offsetURL('search.json'))
    .then(function(response) {
      if (response.status == 200) {
        return response.json().then(function(json) {
          // index main articles
          json.articles.forEach(function(article) {
            fuse.add(article);
          });
          // download collections and index their articles
          return Promise.all(json.collections.map(function(collection) {
            return fetch(offsetURL(collection)).then(function(response) {
              if (response.status === 200) {
                return response.json().then(function(articles) {
                  articles.forEach(function(article) {
                    fuse.add(article);
                  });
                })
              } else {
                return Promise.reject(
                  new Error('Unexpected status from search index request: ' +
                            response.status)
                );
              }
            });
          })).then(function() {
            return fuse;
          });
        });

      } else {
        return Promise.reject(
          new Error('Unexpected status from search index request: ' +
                      response.status)
        );
      }
    });
}

window.document.addEventListener("DOMContentLoaded", function (event) {

  // get search element (bail if we don't have one)
  var searchEl = window.document.getElementById('distill-search');
  if (!searchEl)
    return;

  createFuseIndex()
    .then(function(fuse) {

      // make search box visible
      searchEl.classList.remove('hidden');

      // initialize autocomplete
      var options = {
        autoselect: true,
        hint: false,
        minLength: 2,
      };
      window.autocomplete(searchEl, options, [{
        source: function(query, callback) {
          const searchOptions = {
            isCaseSensitive: false,
            shouldSort: true,
            minMatchCharLength: 2,
            limit: 10,
          };
          var results = fuse.search(query, searchOptions);
          callback(results
            .map(function(result) { return result.item; })
          );
        },
        templates: {
          suggestion: function(suggestion) {
            var img = suggestion.preview && Object.keys(suggestion.preview).length > 0
              ? `<img src="${offsetURL(suggestion.preview)}"</img>`
              : '';
            var html = `
              <div class="search-item">
                <h3>${suggestion.title}</h3>
                <div class="search-item-description">
                  ${suggestion.description || ''}
                </div>
                <div class="search-item-preview">
                  ${img}
                </div>
              </div>
            `;
            return html;
          }
        }
      }]).on('autocomplete:selected', function(event, suggestion) {
        window.location.href = offsetURL(suggestion.path);
      });
      // remove inline display style on autocompleter (we want to
      // manage responsive display via css)
      $('.algolia-autocomplete').css("display", "");
    })
    .catch(function(error) {
      console.log(error);
    });

});

</script>

<style type="text/css">

.nav-search {
  font-size: x-small;
}

/* Algolioa Autocomplete */

.algolia-autocomplete {
  display: inline-block;
  margin-left: 10px;
  vertical-align: sub;
  background-color: white;
  color: black;
  padding: 6px;
  padding-top: 8px;
  padding-bottom: 0;
  border-radius: 6px;
  border: 1px #0F2E3D solid;
  width: 180px;
}


@media screen and (max-width: 768px) {
  .distill-site-nav .algolia-autocomplete {
    display: none;
    visibility: hidden;
  }
  .distill-site-nav.responsive .algolia-autocomplete {
    display: inline-block;
    visibility: visible;
  }
  .distill-site-nav.responsive .algolia-autocomplete .aa-dropdown-menu {
    margin-left: 0;
    width: 400px;
    max-height: 400px;
  }
}

.algolia-autocomplete .aa-input, .algolia-autocomplete .aa-hint {
  width: 90%;
  outline: none;
  border: none;
}

.algolia-autocomplete .aa-hint {
  color: #999;
}
.algolia-autocomplete .aa-dropdown-menu {
  width: 550px;
  max-height: 70vh;
  overflow-x: visible;
  overflow-y: scroll;
  padding: 5px;
  margin-top: 3px;
  margin-left: -150px;
  background-color: #fff;
  border-radius: 5px;
  border: 1px solid #999;
  border-top: none;
}

.algolia-autocomplete .aa-dropdown-menu .aa-suggestion {
  cursor: pointer;
  padding: 5px 4px;
  border-bottom: 1px solid #eee;
}

.algolia-autocomplete .aa-dropdown-menu .aa-suggestion:last-of-type {
  border-bottom: none;
  margin-bottom: 2px;
}

.algolia-autocomplete .aa-dropdown-menu .aa-suggestion .search-item {
  overflow: hidden;
  font-size: 0.8em;
  line-height: 1.4em;
}

.algolia-autocomplete .aa-dropdown-menu .aa-suggestion .search-item h3 {
  font-size: 1rem;
  margin-block-start: 0;
  margin-block-end: 5px;
}

.algolia-autocomplete .aa-dropdown-menu .aa-suggestion .search-item-description {
  display: inline-block;
  overflow: hidden;
  height: 2.8em;
  width: 80%;
  margin-right: 4%;
}

.algolia-autocomplete .aa-dropdown-menu .aa-suggestion .search-item-preview {
  display: inline-block;
  width: 15%;
}

.algolia-autocomplete .aa-dropdown-menu .aa-suggestion .search-item-preview img {
  height: 3em;
  width: auto;
  display: none;
}

.algolia-autocomplete .aa-dropdown-menu .aa-suggestion .search-item-preview img[src] {
  display: initial;
}

.algolia-autocomplete .aa-dropdown-menu .aa-suggestion.aa-cursor {
  background-color: #eee;
}
.algolia-autocomplete .aa-dropdown-menu .aa-suggestion em {
  font-weight: bold;
  font-style: normal;
}

</style>


<!--/radix_placeholder_navigation_in_header-->
<!--radix_placeholder_site_in_header-->
<!--/radix_placeholder_site_in_header-->

<style type="text/css">
body {
  padding-top: 60px;
}
</style>
\ifLuaTeX
  \usepackage{selnolig}  % disable illegal ligatures
\fi

\begin{document}
\maketitle

<!--radix_placeholder_navigation_before_body-->
<header class="header header--fixed" role="banner">
<nav class="distill-site-nav distill-site-header">
<div class="nav-left">
<a href="index.html" class="title">Nguyen Nguyen</a>
</div>
<div class="nav-right">
<a href="index.html">Home</a>
<a href="past.html">Past Course Visualization</a>
<a href="lol.html">LoL Story Telling</a>
<a href="javascript:void(0);" class="nav-toggle">&#9776;</a>
</div>
</nav>
</header>
<!--/radix_placeholder_navigation_before_body-->

<!--radix_placeholder_site_before_body-->
<!--/radix_placeholder_site_before_body-->

\begin{Shaded}
\begin{Highlighting}[]
\FunctionTok{library}\NormalTok{(dplyr)}
\end{Highlighting}
\end{Shaded}

\begin{verbatim}
## 
## Attaching package: 'dplyr'
\end{verbatim}

\begin{verbatim}
## The following objects are masked from 'package:stats':
## 
##     filter, lag
\end{verbatim}

\begin{verbatim}
## The following objects are masked from 'package:base':
## 
##     intersect, setdiff, setequal, union
\end{verbatim}

\begin{Shaded}
\begin{Highlighting}[]
\FunctionTok{library}\NormalTok{(ggplot2)}
\end{Highlighting}
\end{Shaded}

\begin{Shaded}
\begin{Highlighting}[]
\NormalTok{data }\OtherTok{=} \FunctionTok{read.csv}\NormalTok{(}\StringTok{"high\_diamond\_ranked\_10min.csv"}\NormalTok{)}
\end{Highlighting}
\end{Shaded}

\begin{Shaded}
\begin{Highlighting}[]
\NormalTok{data}\SpecialCharTok{$}\NormalTok{winnerGoldDiff }\OtherTok{=} \FunctionTok{ifelse}\NormalTok{(data}\SpecialCharTok{$}\NormalTok{blueWins}\SpecialCharTok{==}\DecValTok{1}\NormalTok{, data}\SpecialCharTok{$}\NormalTok{blueGoldDiff, data}\SpecialCharTok{$}\NormalTok{redGoldDiff)}
\end{Highlighting}
\end{Shaded}

\begin{Shaded}
\begin{Highlighting}[]
\NormalTok{data}\SpecialCharTok{$}\NormalTok{blueAggressiveGame }\OtherTok{=} \FunctionTok{ifelse}\NormalTok{(data}\SpecialCharTok{$}\NormalTok{blueKills}\SpecialCharTok{\textgreater{}=}\DecValTok{10}\NormalTok{, }\StringTok{"aggressive"}\NormalTok{, }\StringTok{"non{-}aggressive"}\NormalTok{)}
\NormalTok{data}\SpecialCharTok{$}\NormalTok{redAggressiveGame }\OtherTok{=} \FunctionTok{ifelse}\NormalTok{(data}\SpecialCharTok{$}\NormalTok{redKills}\SpecialCharTok{\textgreater{}=}\DecValTok{10}\NormalTok{, }\StringTok{"aggressive"}\NormalTok{, }\StringTok{"non{-}aggressive"}\NormalTok{)}
\end{Highlighting}
\end{Shaded}

\begin{Shaded}
\begin{Highlighting}[]
\NormalTok{data}\SpecialCharTok{$}\NormalTok{winnderExpDiff }\OtherTok{=} \FunctionTok{ifelse}\NormalTok{(data}\SpecialCharTok{$}\NormalTok{blueWins}\SpecialCharTok{==}\DecValTok{1}\NormalTok{, data}\SpecialCharTok{$}\NormalTok{blueExperienceDiff, data}\SpecialCharTok{$}\NormalTok{redExperienceDiff)}
\end{Highlighting}
\end{Shaded}

\begin{Shaded}
\begin{Highlighting}[]
\FunctionTok{library}\NormalTok{(ggplot2)}
\end{Highlighting}
\end{Shaded}

\begin{Shaded}
\begin{Highlighting}[]
\FunctionTok{colnames}\NormalTok{(data)}
\end{Highlighting}
\end{Shaded}

\begin{verbatim}
##  [1] "gameId"                       "blueWins"                    
##  [3] "blueWardsPlaced"              "blueWardsDestroyed"          
##  [5] "blueFirstBlood"               "blueKills"                   
##  [7] "blueDeaths"                   "blueAssists"                 
##  [9] "blueEliteMonsters"            "blueDragons"                 
## [11] "blueHeralds"                  "blueTowersDestroyed"         
## [13] "blueTotalGold"                "blueAvgLevel"                
## [15] "blueTotalExperience"          "blueTotalMinionsKilled"      
## [17] "blueTotalJungleMinionsKilled" "blueGoldDiff"                
## [19] "blueExperienceDiff"           "blueCSPerMin"                
## [21] "blueGoldPerMin"               "redWardsPlaced"              
## [23] "redWardsDestroyed"            "redFirstBlood"               
## [25] "redKills"                     "redDeaths"                   
## [27] "redAssists"                   "redEliteMonsters"            
## [29] "redDragons"                   "redHeralds"                  
## [31] "redTowersDestroyed"           "redTotalGold"                
## [33] "redAvgLevel"                  "redTotalExperience"          
## [35] "redTotalMinionsKilled"        "redTotalJungleMinionsKilled" 
## [37] "redGoldDiff"                  "redExperienceDiff"           
## [39] "redCSPerMin"                  "redGoldPerMin"               
## [41] "winnerGoldDiff"               "blueAggressiveGame"          
## [43] "redAggressiveGame"            "winnderExpDiff"
\end{verbatim}

\hypertarget{finding-the-most-important-winning-factors-for-blue-side}{%
\subsection{Finding the most important winning factors for blue
side}\label{finding-the-most-important-winning-factors-for-blue-side}}

\begin{Shaded}
\begin{Highlighting}[]
\NormalTok{keep\_cols }\OtherTok{\textless{}{-}} \FunctionTok{grep}\NormalTok{(}\StringTok{\textquotesingle{}red\textquotesingle{}}\NormalTok{, }\FunctionTok{names}\NormalTok{(data), }\AttributeTok{invert =} \ConstantTok{TRUE}\NormalTok{)}
\NormalTok{blue\_data }\OtherTok{=}\NormalTok{ data[, keep\_cols]}
\end{Highlighting}
\end{Shaded}

\begin{Shaded}
\begin{Highlighting}[]
\NormalTok{numeric\_cols }\OtherTok{\textless{}{-}} \FunctionTok{sapply}\NormalTok{(blue\_data, is.numeric)}
\NormalTok{cor\_blue }\OtherTok{=}\NormalTok{ blue\_data[, numeric\_cols]}
\NormalTok{corr }\OtherTok{\textless{}{-}} \FunctionTok{round}\NormalTok{(}\FunctionTok{cor}\NormalTok{(cor\_blue), }\DecValTok{2}\NormalTok{)}
\NormalTok{data}\SpecialCharTok{$}\NormalTok{redWins }\OtherTok{=} \FunctionTok{ifelse}\NormalTok{(data}\SpecialCharTok{$}\NormalTok{blueWins }\SpecialCharTok{==} \DecValTok{0}\NormalTok{, }\DecValTok{1}\NormalTok{, }\DecValTok{0}\NormalTok{)}
\NormalTok{blue\_winning\_factors }\OtherTok{=} \FunctionTok{names}\NormalTok{(}\FunctionTok{sort}\NormalTok{(corr[}\FunctionTok{c}\NormalTok{(}\StringTok{"blueWins"}\NormalTok{),], }\AttributeTok{decreasing=}\NormalTok{T)[}\DecValTok{1}\SpecialCharTok{:}\DecValTok{10}\NormalTok{])}
\end{Highlighting}
\end{Shaded}

\begin{Shaded}
\begin{Highlighting}[]
\FunctionTok{library}\NormalTok{(ggcorrplot)}
\NormalTok{corr }\OtherTok{=}\NormalTok{ corr[blue\_winning\_factors,blue\_winning\_factors]}
\FunctionTok{ggcorrplot}\NormalTok{(}\AttributeTok{corr =}\NormalTok{ corr)}
\end{Highlighting}
\end{Shaded}

\includegraphics{lol_files/figure-latex/unnamed-chunk-10-1.pdf} So, the
most important factors are Gold, Experience, and Kills

\hypertarget{understanding-gold}{%
\subsubsection{Understanding Gold}\label{understanding-gold}}

In League of Legends, gold can be obtained by:

\begin{enumerate}
\def\labelenumi{\arabic{enumi}.}
\item
  Killing minions and monsters
\item
  Destroying towers
\item
  (Assisting) Killing enemy champions
\end{enumerate}

We will see which of the following matters the most:

\begin{Shaded}
\begin{Highlighting}[]
\FunctionTok{colnames}\NormalTok{(data)}
\end{Highlighting}
\end{Shaded}

\begin{verbatim}
##  [1] "gameId"                       "blueWins"                    
##  [3] "blueWardsPlaced"              "blueWardsDestroyed"          
##  [5] "blueFirstBlood"               "blueKills"                   
##  [7] "blueDeaths"                   "blueAssists"                 
##  [9] "blueEliteMonsters"            "blueDragons"                 
## [11] "blueHeralds"                  "blueTowersDestroyed"         
## [13] "blueTotalGold"                "blueAvgLevel"                
## [15] "blueTotalExperience"          "blueTotalMinionsKilled"      
## [17] "blueTotalJungleMinionsKilled" "blueGoldDiff"                
## [19] "blueExperienceDiff"           "blueCSPerMin"                
## [21] "blueGoldPerMin"               "redWardsPlaced"              
## [23] "redWardsDestroyed"            "redFirstBlood"               
## [25] "redKills"                     "redDeaths"                   
## [27] "redAssists"                   "redEliteMonsters"            
## [29] "redDragons"                   "redHeralds"                  
## [31] "redTowersDestroyed"           "redTotalGold"                
## [33] "redAvgLevel"                  "redTotalExperience"          
## [35] "redTotalMinionsKilled"        "redTotalJungleMinionsKilled" 
## [37] "redGoldDiff"                  "redExperienceDiff"           
## [39] "redCSPerMin"                  "redGoldPerMin"               
## [41] "winnerGoldDiff"               "blueAggressiveGame"          
## [43] "redAggressiveGame"            "winnderExpDiff"              
## [45] "redWins"
\end{verbatim}

\begin{Shaded}
\begin{Highlighting}[]
\NormalTok{blue\_gold\_factors }\OtherTok{=}\NormalTok{ data[,}\FunctionTok{c}\NormalTok{(}\StringTok{"blueTotalJungleMinionsKilled"}\NormalTok{, }
                            \StringTok{"blueKills"}\NormalTok{,}
                            \StringTok{"blueTotalMinionsKilled"}\NormalTok{,}
                            \StringTok{"blueTowersDestroyed"}\NormalTok{,}
                            \StringTok{"blueTotalGold"}
\NormalTok{                            )]}

\NormalTok{corr }\OtherTok{=} \FunctionTok{round}\NormalTok{(}\FunctionTok{cor}\NormalTok{(blue\_gold\_factors), }\DecValTok{2}\NormalTok{)}
\FunctionTok{ggcorrplot}\NormalTok{(}\AttributeTok{corr =}\NormalTok{ corr)}
\end{Highlighting}
\end{Shaded}

\includegraphics{lol_files/figure-latex/unnamed-chunk-12-1.pdf}

We see that the three strongest factors for gold gain are, in descending
order, (1) kills, (2) towers, and (3) minions.

In League, kills are very important: They create opportunities for our
team to (1) gain gold (2) denying enemy from killing minions and (3) set
up for destroying turrets. As such, let's see how kills affect other
factors.

\begin{Shaded}
\begin{Highlighting}[]
\NormalTok{data}\SpecialCharTok{$}\NormalTok{blueWins }\OtherTok{=} \FunctionTok{factor}\NormalTok{(data}\SpecialCharTok{$}\NormalTok{blueWins)}
\end{Highlighting}
\end{Shaded}

\begin{Shaded}
\begin{Highlighting}[]
\FunctionTok{ggplot}\NormalTok{(data, }\FunctionTok{aes}\NormalTok{(}\AttributeTok{x=}\NormalTok{blueWins,}\AttributeTok{fill=}\NormalTok{blueAggressiveGame)) }\SpecialCharTok{+} 
  \FunctionTok{geom\_bar}\NormalTok{(}\AttributeTok{position=}\StringTok{\textquotesingle{}dodge\textquotesingle{}}\NormalTok{)}
\end{Highlighting}
\end{Shaded}

\includegraphics{lol_files/figure-latex/unnamed-chunk-14-1.pdf} So, if
Blue plays aggressively, there is more chance that they win than when
they play passively.

\begin{Shaded}
\begin{Highlighting}[]
\FunctionTok{ggplot}\NormalTok{(data, }\FunctionTok{aes}\NormalTok{(}\AttributeTok{y=}\NormalTok{blueTotalGold, }\AttributeTok{x=}\NormalTok{blueAggressiveGame, }\AttributeTok{fill=}\NormalTok{blueAggressiveGame)) }\SpecialCharTok{+} 
  \FunctionTok{geom\_violin}\NormalTok{(}\AttributeTok{scale=}\StringTok{\textquotesingle{}width\textquotesingle{}}\NormalTok{, }\AttributeTok{alpha=}\FloatTok{0.5}\NormalTok{, ) }\SpecialCharTok{+}
  \FunctionTok{guides}\NormalTok{(}\AttributeTok{fill=}\StringTok{\textquotesingle{}none\textquotesingle{}}\NormalTok{) }\SpecialCharTok{+} 
  \FunctionTok{geom\_boxplot}\NormalTok{(}\AttributeTok{width=}\FloatTok{0.25}\NormalTok{)}
\end{Highlighting}
\end{Shaded}

\includegraphics{lol_files/figure-latex/unnamed-chunk-15-1.pdf} So, if
Blue plays aggressively, there is more chance that they get more gold.

We see that kills are independent of minion. Let's take a closer look:

\begin{Shaded}
\begin{Highlighting}[]
\FunctionTok{ggplot}\NormalTok{(data, }\FunctionTok{aes}\NormalTok{(}\AttributeTok{x=}\NormalTok{blueKills, }\AttributeTok{y=}\NormalTok{blueTotalMinionsKilled)) }\SpecialCharTok{+} 
  \FunctionTok{geom\_point}\NormalTok{() }\SpecialCharTok{+}
  \FunctionTok{geom\_smooth}\NormalTok{(}\AttributeTok{method=}\StringTok{\textquotesingle{}lm\textquotesingle{}}\NormalTok{)}
\end{Highlighting}
\end{Shaded}

\begin{verbatim}
## `geom_smooth()` using formula = 'y ~ x'
\end{verbatim}

\includegraphics{lol_files/figure-latex/unnamed-chunk-16-1.pdf}

Let's see how this changes in aggressive games vs non-aggressive games

\begin{Shaded}
\begin{Highlighting}[]
\FunctionTok{ggplot}\NormalTok{(data, }\FunctionTok{aes}\NormalTok{(}\AttributeTok{y=}\NormalTok{blueTotalMinionsKilled, }\AttributeTok{x=}\NormalTok{blueAggressiveGame, }\AttributeTok{fill=}\NormalTok{blueAggressiveGame)) }\SpecialCharTok{+} 
  \FunctionTok{geom\_violin}\NormalTok{(}\AttributeTok{scale=}\StringTok{\textquotesingle{}width\textquotesingle{}}\NormalTok{, }\AttributeTok{alpha=}\FloatTok{0.5}\NormalTok{, ) }\SpecialCharTok{+}
  \FunctionTok{guides}\NormalTok{(}\AttributeTok{fill=}\StringTok{\textquotesingle{}none\textquotesingle{}}\NormalTok{) }\SpecialCharTok{+} 
  \FunctionTok{geom\_boxplot}\NormalTok{(}\AttributeTok{width=}\FloatTok{0.25}\NormalTok{)}
\end{Highlighting}
\end{Shaded}

\includegraphics{lol_files/figure-latex/unnamed-chunk-17-1.pdf}

In League, the jungler role is crucial: they set up ganks to give kills
to their team. However, in doing so, they are unable to kill the jungle
camps. We see a similar pattern in the dataset:

\begin{Shaded}
\begin{Highlighting}[]
\FunctionTok{ggplot}\NormalTok{(data, }\FunctionTok{aes}\NormalTok{(}\AttributeTok{x=}\NormalTok{blueKills, }\AttributeTok{y=}\NormalTok{blueTotalJungleMinionsKilled)) }\SpecialCharTok{+} 
  \FunctionTok{geom\_point}\NormalTok{() }\SpecialCharTok{+}
  \FunctionTok{geom\_smooth}\NormalTok{(}\AttributeTok{method=}\StringTok{\textquotesingle{}lm\textquotesingle{}}\NormalTok{)}
\end{Highlighting}
\end{Shaded}

\begin{verbatim}
## `geom_smooth()` using formula = 'y ~ x'
\end{verbatim}

\includegraphics{lol_files/figure-latex/unnamed-chunk-18-1.pdf}

\begin{Shaded}
\begin{Highlighting}[]
\FunctionTok{ggplot}\NormalTok{(data, }\FunctionTok{aes}\NormalTok{(}\AttributeTok{y=}\NormalTok{blueTotalJungleMinionsKilled, }\AttributeTok{x=}\NormalTok{blueAggressiveGame, }\AttributeTok{fill=}\NormalTok{blueAggressiveGame)) }\SpecialCharTok{+} 
  \FunctionTok{geom\_violin}\NormalTok{(}\AttributeTok{scale=}\StringTok{\textquotesingle{}width\textquotesingle{}}\NormalTok{, }\AttributeTok{alpha=}\FloatTok{0.5}\NormalTok{, ) }\SpecialCharTok{+}
  \FunctionTok{guides}\NormalTok{(}\AttributeTok{fill=}\StringTok{\textquotesingle{}none\textquotesingle{}}\NormalTok{) }\SpecialCharTok{+} 
  \FunctionTok{geom\_boxplot}\NormalTok{(}\AttributeTok{width=}\FloatTok{0.25}\NormalTok{)}
\end{Highlighting}
\end{Shaded}

\includegraphics{lol_files/figure-latex/unnamed-chunk-19-1.pdf}

\begin{Shaded}
\begin{Highlighting}[]
\FunctionTok{ggplot}\NormalTok{(data, }\FunctionTok{aes}\NormalTok{(}\AttributeTok{y=}\NormalTok{blueTotalJungleMinionsKilled, }\AttributeTok{x=}\NormalTok{blueAggressiveGame, }\AttributeTok{fill=}\NormalTok{blueAggressiveGame)) }\SpecialCharTok{+} 
  \FunctionTok{geom\_violin}\NormalTok{(}\AttributeTok{scale=}\StringTok{\textquotesingle{}width\textquotesingle{}}\NormalTok{, }\AttributeTok{alpha=}\FloatTok{0.5}\NormalTok{, ) }\SpecialCharTok{+}
  \FunctionTok{guides}\NormalTok{(}\AttributeTok{fill=}\StringTok{\textquotesingle{}none\textquotesingle{}}\NormalTok{) }\SpecialCharTok{+} 
  \FunctionTok{geom\_boxplot}\NormalTok{(}\AttributeTok{width=}\FloatTok{0.25}\NormalTok{) }\SpecialCharTok{+}
  \FunctionTok{facet\_wrap}\NormalTok{(}\SpecialCharTok{\textasciitilde{}}\NormalTok{blueWins)}
\end{Highlighting}
\end{Shaded}

\includegraphics{lol_files/figure-latex/unnamed-chunk-20-1.pdf} As
expected, in games where blue wins, the jungler, on average, gets more
minions.

Finally, let's see how kills affect towers taken:

\begin{Shaded}
\begin{Highlighting}[]
\FunctionTok{ggplot}\NormalTok{(data, }\FunctionTok{aes}\NormalTok{(}\AttributeTok{x=}\NormalTok{blueTowersDestroyed, }\AttributeTok{color=}\NormalTok{blueAggressiveGame)) }\SpecialCharTok{+} 
  \FunctionTok{geom\_density}\NormalTok{()  }\SpecialCharTok{+}
  \FunctionTok{guides}\NormalTok{(}\AttributeTok{fill=}\StringTok{\textquotesingle{}none\textquotesingle{}}\NormalTok{) }\SpecialCharTok{+} 
  \FunctionTok{facet\_wrap}\NormalTok{(}\SpecialCharTok{\textasciitilde{}}\NormalTok{blueWins)}
\end{Highlighting}
\end{Shaded}

\includegraphics{lol_files/figure-latex/unnamed-chunk-21-1.pdf}

\begin{Shaded}
\begin{Highlighting}[]
\FunctionTok{ggplot}\NormalTok{(data, }\FunctionTok{aes}\NormalTok{(}\AttributeTok{x=}\NormalTok{blueTowersDestroyed, }\AttributeTok{color=}\NormalTok{blueAggressiveGame)) }\SpecialCharTok{+} 
  \FunctionTok{geom\_density}\NormalTok{()  }\SpecialCharTok{+}
  \FunctionTok{guides}\NormalTok{(}\AttributeTok{fill=}\StringTok{\textquotesingle{}none\textquotesingle{}}\NormalTok{)}
\end{Highlighting}
\end{Shaded}

\includegraphics{lol_files/figure-latex/unnamed-chunk-22-1.pdf}

In general, if blue is playing more aggressive, they could get more
towers.

\hypertarget{experience}{%
\subsection{Experience}\label{experience}}

In a similar fashion to gold, in league of legends, one could get
experience by

\begin{enumerate}
\def\labelenumi{\arabic{enumi}.}
\item
  Killing minions and monsters
\item
  (Assisting) Killing enemy champions
\end{enumerate}

\begin{Shaded}
\begin{Highlighting}[]
\NormalTok{blue\_exp\_factors }\OtherTok{=}\NormalTok{ data[,}\FunctionTok{c}\NormalTok{(}\StringTok{"blueTotalJungleMinionsKilled"}\NormalTok{, }
                            \StringTok{"blueKills"}\NormalTok{,}
                            \StringTok{"blueTotalMinionsKilled"}\NormalTok{,}
                            \StringTok{"blueTotalExperience"}
\NormalTok{                            )]}

\NormalTok{corr }\OtherTok{=} \FunctionTok{round}\NormalTok{(}\FunctionTok{cor}\NormalTok{(blue\_exp\_factors), }\DecValTok{2}\NormalTok{)}
\FunctionTok{ggcorrplot}\NormalTok{(}\AttributeTok{corr =}\NormalTok{ corr)}
\end{Highlighting}
\end{Shaded}

\includegraphics{lol_files/figure-latex/unnamed-chunk-23-1.pdf}

We see that total experience is affected by minion kills the most,
followed by champion kills, then jungle minions.

Let us see how minion kills

\begin{Shaded}
\begin{Highlighting}[]
\FunctionTok{ggplot}\NormalTok{(data, }\FunctionTok{aes}\NormalTok{(}\AttributeTok{y=}\NormalTok{blueTotalExperience, }\AttributeTok{x=}\NormalTok{blueAggressiveGame, }\AttributeTok{fill=}\NormalTok{blueAggressiveGame)) }\SpecialCharTok{+} 
  \FunctionTok{geom\_violin}\NormalTok{(}\AttributeTok{scale=}\StringTok{\textquotesingle{}width\textquotesingle{}}\NormalTok{, }\AttributeTok{alpha=}\FloatTok{0.5}\NormalTok{, ) }\SpecialCharTok{+}
  \FunctionTok{guides}\NormalTok{(}\AttributeTok{fill=}\StringTok{\textquotesingle{}none\textquotesingle{}}\NormalTok{) }\SpecialCharTok{+} 
  \FunctionTok{geom\_boxplot}\NormalTok{(}\AttributeTok{width=}\FloatTok{0.25}\NormalTok{)}
\end{Highlighting}
\end{Shaded}

\includegraphics{lol_files/figure-latex/unnamed-chunk-24-1.pdf} In a
similar fashion, playing aggressively helps gaining experiences.

\hypertarget{red}{%
\subsection{Red}\label{red}}

\begin{Shaded}
\begin{Highlighting}[]
\NormalTok{keep\_cols }\OtherTok{\textless{}{-}} \FunctionTok{grep}\NormalTok{(}\StringTok{\textquotesingle{}blue\textquotesingle{}}\NormalTok{, }\FunctionTok{names}\NormalTok{(data), }\AttributeTok{invert =} \ConstantTok{TRUE}\NormalTok{)}
\NormalTok{red\_data }\OtherTok{=}\NormalTok{ data[, keep\_cols]}\CommentTok{\#[,keep\_cols]}
\end{Highlighting}
\end{Shaded}

\begin{Shaded}
\begin{Highlighting}[]
\NormalTok{numeric\_cols }\OtherTok{\textless{}{-}} \FunctionTok{sapply}\NormalTok{(red\_data, is.numeric)}
\NormalTok{cor\_red }\OtherTok{=}\NormalTok{ red\_data[, numeric\_cols]}
\NormalTok{corr }\OtherTok{\textless{}{-}} \FunctionTok{round}\NormalTok{(}\FunctionTok{cor}\NormalTok{(cor\_red), }\DecValTok{2}\NormalTok{)}
\NormalTok{red\_winning\_factors }\OtherTok{=} \FunctionTok{names}\NormalTok{(}\FunctionTok{sort}\NormalTok{(corr[}\FunctionTok{c}\NormalTok{(}\StringTok{"redWins"}\NormalTok{),], }\AttributeTok{decreasing=}\NormalTok{T)[}\DecValTok{1}\SpecialCharTok{:}\DecValTok{10}\NormalTok{])}
\NormalTok{numeric\_cols }\OtherTok{\textless{}{-}} \FunctionTok{sapply}\NormalTok{(red\_data, is.numeric)}
\NormalTok{corr }\OtherTok{=}\NormalTok{ corr[red\_winning\_factors,red\_winning\_factors]}
\FunctionTok{ggcorrplot}\NormalTok{(}\AttributeTok{corr =}\NormalTok{ corr)}
\end{Highlighting}
\end{Shaded}

\includegraphics{lol_files/figure-latex/unnamed-chunk-26-1.pdf}

\begin{Shaded}
\begin{Highlighting}[]
\NormalTok{data}\SpecialCharTok{$}\NormalTok{redWins }\OtherTok{=} \FunctionTok{factor}\NormalTok{(data}\SpecialCharTok{$}\NormalTok{redWins)}
\end{Highlighting}
\end{Shaded}

\begin{Shaded}
\begin{Highlighting}[]
\FunctionTok{ggplot}\NormalTok{(data, }\FunctionTok{aes}\NormalTok{(}\AttributeTok{x=}\NormalTok{redWins,}\AttributeTok{fill=}\NormalTok{redAggressiveGame)) }\SpecialCharTok{+} 
  \FunctionTok{geom\_bar}\NormalTok{(}\AttributeTok{position=}\StringTok{\textquotesingle{}dodge\textquotesingle{}}\NormalTok{)}
\end{Highlighting}
\end{Shaded}

\includegraphics{lol_files/figure-latex/unnamed-chunk-28-1.pdf} So, if
red plays aggressively, there is more chance that they win than when
they play passively.

\begin{Shaded}
\begin{Highlighting}[]
\FunctionTok{ggplot}\NormalTok{(data, }\FunctionTok{aes}\NormalTok{(}\AttributeTok{y=}\NormalTok{redTotalGold, }\AttributeTok{x=}\NormalTok{redAggressiveGame, }\AttributeTok{fill=}\NormalTok{redAggressiveGame)) }\SpecialCharTok{+} 
  \FunctionTok{geom\_violin}\NormalTok{(}\AttributeTok{scale=}\StringTok{\textquotesingle{}width\textquotesingle{}}\NormalTok{, }\AttributeTok{alpha=}\FloatTok{0.5}\NormalTok{, ) }\SpecialCharTok{+}
  \FunctionTok{guides}\NormalTok{(}\AttributeTok{fill=}\StringTok{\textquotesingle{}none\textquotesingle{}}\NormalTok{) }\SpecialCharTok{+} 
  \FunctionTok{geom\_boxplot}\NormalTok{(}\AttributeTok{width=}\FloatTok{0.25}\NormalTok{)}
\end{Highlighting}
\end{Shaded}

\includegraphics{lol_files/figure-latex/unnamed-chunk-29-1.pdf} So, if
red plays aggressively, there is more chance that they get more gold.

We see that kills are independent of minion. Let's take a closer look:

\begin{Shaded}
\begin{Highlighting}[]
\FunctionTok{ggplot}\NormalTok{(data, }\FunctionTok{aes}\NormalTok{(}\AttributeTok{x=}\NormalTok{redKills, }\AttributeTok{y=}\NormalTok{redTotalMinionsKilled)) }\SpecialCharTok{+} 
  \FunctionTok{geom\_point}\NormalTok{() }\SpecialCharTok{+}
  \FunctionTok{geom\_smooth}\NormalTok{(}\AttributeTok{method=}\StringTok{\textquotesingle{}lm\textquotesingle{}}\NormalTok{)}
\end{Highlighting}
\end{Shaded}

\begin{verbatim}
## `geom_smooth()` using formula = 'y ~ x'
\end{verbatim}

\includegraphics{lol_files/figure-latex/unnamed-chunk-30-1.pdf}

Let's see how this changes in aggressive games vs non-aggressive games

\begin{Shaded}
\begin{Highlighting}[]
\FunctionTok{ggplot}\NormalTok{(data, }\FunctionTok{aes}\NormalTok{(}\AttributeTok{y=}\NormalTok{redTotalMinionsKilled, }\AttributeTok{x=}\NormalTok{redAggressiveGame, }\AttributeTok{fill=}\NormalTok{redAggressiveGame)) }\SpecialCharTok{+} 
  \FunctionTok{geom\_violin}\NormalTok{(}\AttributeTok{scale=}\StringTok{\textquotesingle{}width\textquotesingle{}}\NormalTok{, }\AttributeTok{alpha=}\FloatTok{0.5}\NormalTok{, ) }\SpecialCharTok{+}
  \FunctionTok{guides}\NormalTok{(}\AttributeTok{fill=}\StringTok{\textquotesingle{}none\textquotesingle{}}\NormalTok{) }\SpecialCharTok{+} 
  \FunctionTok{geom\_boxplot}\NormalTok{(}\AttributeTok{width=}\FloatTok{0.25}\NormalTok{)}
\end{Highlighting}
\end{Shaded}

\includegraphics{lol_files/figure-latex/unnamed-chunk-31-1.pdf}

In League, the jungler role is crucial: they set up ganks to give kills
to their team. However, in doing so, they are unable to kill the jungle
camps. We see a similar pattern in the dataset:

\begin{Shaded}
\begin{Highlighting}[]
\FunctionTok{ggplot}\NormalTok{(data, }\FunctionTok{aes}\NormalTok{(}\AttributeTok{x=}\NormalTok{redKills, }\AttributeTok{y=}\NormalTok{redTotalJungleMinionsKilled)) }\SpecialCharTok{+} 
  \FunctionTok{geom\_point}\NormalTok{() }\SpecialCharTok{+}
  \FunctionTok{geom\_smooth}\NormalTok{(}\AttributeTok{method=}\StringTok{\textquotesingle{}lm\textquotesingle{}}\NormalTok{)}
\end{Highlighting}
\end{Shaded}

\begin{verbatim}
## `geom_smooth()` using formula = 'y ~ x'
\end{verbatim}

\includegraphics{lol_files/figure-latex/unnamed-chunk-32-1.pdf}

\begin{Shaded}
\begin{Highlighting}[]
\FunctionTok{ggplot}\NormalTok{(data, }\FunctionTok{aes}\NormalTok{(}\AttributeTok{y=}\NormalTok{redTotalJungleMinionsKilled, }\AttributeTok{x=}\NormalTok{redAggressiveGame, }\AttributeTok{fill=}\NormalTok{redAggressiveGame)) }\SpecialCharTok{+} 
  \FunctionTok{geom\_violin}\NormalTok{(}\AttributeTok{scale=}\StringTok{\textquotesingle{}width\textquotesingle{}}\NormalTok{, }\AttributeTok{alpha=}\FloatTok{0.5}\NormalTok{, ) }\SpecialCharTok{+}
  \FunctionTok{guides}\NormalTok{(}\AttributeTok{fill=}\StringTok{\textquotesingle{}none\textquotesingle{}}\NormalTok{) }\SpecialCharTok{+} 
  \FunctionTok{geom\_boxplot}\NormalTok{(}\AttributeTok{width=}\FloatTok{0.25}\NormalTok{)}
\end{Highlighting}
\end{Shaded}

\includegraphics{lol_files/figure-latex/unnamed-chunk-33-1.pdf}

\begin{Shaded}
\begin{Highlighting}[]
\FunctionTok{ggplot}\NormalTok{(data, }\FunctionTok{aes}\NormalTok{(}\AttributeTok{y=}\NormalTok{redTotalJungleMinionsKilled, }\AttributeTok{x=}\NormalTok{redAggressiveGame, }\AttributeTok{fill=}\NormalTok{redAggressiveGame)) }\SpecialCharTok{+} 
  \FunctionTok{geom\_violin}\NormalTok{(}\AttributeTok{scale=}\StringTok{\textquotesingle{}width\textquotesingle{}}\NormalTok{, }\AttributeTok{alpha=}\FloatTok{0.5}\NormalTok{, ) }\SpecialCharTok{+}
  \FunctionTok{guides}\NormalTok{(}\AttributeTok{fill=}\StringTok{\textquotesingle{}none\textquotesingle{}}\NormalTok{) }\SpecialCharTok{+} 
  \FunctionTok{geom\_boxplot}\NormalTok{(}\AttributeTok{width=}\FloatTok{0.25}\NormalTok{) }\SpecialCharTok{+}
  \FunctionTok{facet\_wrap}\NormalTok{(}\SpecialCharTok{\textasciitilde{}}\NormalTok{redWins)}
\end{Highlighting}
\end{Shaded}

\includegraphics{lol_files/figure-latex/unnamed-chunk-34-1.pdf} As
expected, in games where red wins, the jungler, on average, gets more
minions.

Finally, let's see how kills affect towers taken:

\begin{Shaded}
\begin{Highlighting}[]
\FunctionTok{ggplot}\NormalTok{(data, }\FunctionTok{aes}\NormalTok{(}\AttributeTok{x=}\NormalTok{redTowersDestroyed, }\AttributeTok{color=}\NormalTok{redAggressiveGame)) }\SpecialCharTok{+} 
  \FunctionTok{geom\_density}\NormalTok{()  }\SpecialCharTok{+}
  \FunctionTok{guides}\NormalTok{(}\AttributeTok{fill=}\StringTok{\textquotesingle{}none\textquotesingle{}}\NormalTok{) }\SpecialCharTok{+} 
  \FunctionTok{facet\_wrap}\NormalTok{(}\SpecialCharTok{\textasciitilde{}}\NormalTok{redWins)}
\end{Highlighting}
\end{Shaded}

\includegraphics{lol_files/figure-latex/unnamed-chunk-35-1.pdf}

\begin{Shaded}
\begin{Highlighting}[]
\FunctionTok{ggplot}\NormalTok{(data, }\FunctionTok{aes}\NormalTok{(}\AttributeTok{x=}\NormalTok{redTowersDestroyed, }\AttributeTok{color=}\NormalTok{redAggressiveGame)) }\SpecialCharTok{+} 
  \FunctionTok{geom\_density}\NormalTok{()  }\SpecialCharTok{+}
  \FunctionTok{guides}\NormalTok{(}\AttributeTok{fill=}\StringTok{\textquotesingle{}none\textquotesingle{}}\NormalTok{)}
\end{Highlighting}
\end{Shaded}

\includegraphics{lol_files/figure-latex/unnamed-chunk-36-1.pdf}

In general, if red is playing more aggressive, they could get more
towers.

\hypertarget{experience-1}{%
\subsection{Experience}\label{experience-1}}

In a similar fashion to gold, in league of legends, one could get
experience by

\begin{enumerate}
\def\labelenumi{\arabic{enumi}.}
\item
  Killing minions and monsters
\item
  (Assisting) Killing enemy champions
\end{enumerate}

\begin{Shaded}
\begin{Highlighting}[]
\NormalTok{red\_exp\_factors }\OtherTok{=}\NormalTok{ data[,}\FunctionTok{c}\NormalTok{(}\StringTok{"redTotalJungleMinionsKilled"}\NormalTok{, }
                            \StringTok{"redKills"}\NormalTok{,}
                            \StringTok{"redTotalMinionsKilled"}\NormalTok{,}
                            \StringTok{"redTotalExperience"}
\NormalTok{                            )]}

\NormalTok{corr }\OtherTok{=} \FunctionTok{round}\NormalTok{(}\FunctionTok{cor}\NormalTok{(red\_exp\_factors), }\DecValTok{2}\NormalTok{)}
\FunctionTok{ggcorrplot}\NormalTok{(}\AttributeTok{corr =}\NormalTok{ corr)}
\end{Highlighting}
\end{Shaded}

\includegraphics{lol_files/figure-latex/unnamed-chunk-37-1.pdf}

We see that total experience is affected by minion kills the most,
followed by champion kills, then jungle minions.

Let us see how minion kills

\begin{Shaded}
\begin{Highlighting}[]
\FunctionTok{ggplot}\NormalTok{(data, }\FunctionTok{aes}\NormalTok{(}\AttributeTok{y=}\NormalTok{redTotalExperience, }\AttributeTok{x=}\NormalTok{redAggressiveGame, }\AttributeTok{fill=}\NormalTok{redAggressiveGame)) }\SpecialCharTok{+} 
  \FunctionTok{geom\_violin}\NormalTok{(}\AttributeTok{scale=}\StringTok{\textquotesingle{}width\textquotesingle{}}\NormalTok{, }\AttributeTok{alpha=}\FloatTok{0.5}\NormalTok{, ) }\SpecialCharTok{+}
  \FunctionTok{guides}\NormalTok{(}\AttributeTok{fill=}\StringTok{\textquotesingle{}none\textquotesingle{}}\NormalTok{) }\SpecialCharTok{+} 
  \FunctionTok{geom\_boxplot}\NormalTok{(}\AttributeTok{width=}\FloatTok{0.25}\NormalTok{)}
\end{Highlighting}
\end{Shaded}

\includegraphics{lol_files/figure-latex/unnamed-chunk-38-1.pdf} In a
similar fashion, playing aggressively helps gaining experiences.

<!--radix_placeholder_site_after_body-->
<!--/radix_placeholder_site_after_body-->

<!--radix_placeholder_navigation_after_body-->
<!--/radix_placeholder_navigation_after_body-->

\end{document}
